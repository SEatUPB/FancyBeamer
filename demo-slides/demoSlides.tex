\documentclass[
	aspectratio=169, % default is 43
	8pt, % font size, default is 11pt
	%handout, % handout mode without animations, comment out to add animations
	%nosectionframes, % disable automatic frames at the begin of each section (default: sectiontitleslides in beamer mode and sectionoverviews in handout mode)
	%sectiontitleslides, % enable an automatic section title slide at the begin of each section
	%sectionoverviews, % enable an automatic section overview at the begin of each section
	%uniqueslidenumber, % will uniquely identify pages with overlays by a little suffix
	%darkmode, % switch to dark mode (do not use for presentation with a projector)
]{beamer}
\usepackage{../fancybeamer} % use the fancy beamer package

%\usepackage[ngerman]{babel} % use this line for slides in German

%\includeonlyframes{current} % default mechanism of beamer to include only labeled frames, can be used for debugging or drafting

% \AtEndPreamble{\setpaths{{pics/}{../pics/}}} % specify custom paths to your pictures

\title[Short Title]{This is the Long Title} % short title is used for the slide footer but optional
\subtitle[Short Subtitle]{This is the Long Subtitle} % subtitles are optional at all
\author[Short Author]{Long Author} % short author title is used for the slide footer but optional
\date{\today} % use a particular date here if needed

\fancylogos{example-image-a,example-image-b,example-image-c} % define logos that are spread across the bottom of the title slide

\begin{document}

%\maketitle[] % title page without a picture
%\maketitle[example-image] % ... with title picture
\maketitle[example-image][50] % ... with optional parameter to move it upwards

\section{Slide Contents}

\subsection{Lists and Numbered Lists}
\begin{frame}{\insertsubsection}
	\begin{mycolumns}
		Lists can be nested to a depth of three:
		\begin{itemize}
			\item Item on the first level
			\item Another item on the first level
			\begin{itemize}
				\item Item on the second level
				\item Another item on the second level
			\begin{itemize}
			\item Item on the third level
			\item Another item on the third level
			\end{itemize}
		\end{itemize}
	\end{itemize}
	\mynextcolumn
	Numbered lists can be nested to a depth of three:
	\begin{enumerate}
		\item Item on the first level
		\item Another item on the first level
		\begin{enumerate}
			\item Item on the second level
			\item Another item on the second level
				\begin{enumerate}
					\item Item on the third level
					\item Another item on the third level
				\end{enumerate}
			\end{enumerate}
		\end{enumerate}
	\end{mycolumns}
\end{frame}

\subsection{Colored Boxes}
\begin{frame}{\insertsubsection}
	Normal versions:
	\begin{mycolumns}[columns=3]
		\begin{definition}{A Definition}
			This is a definition.
		\end{definition}
	\mynextcolumn
		\begin{example}{An Example}
			This is an example.
		\end{example}
	\mynextcolumn
		\begin{note}{A Note}
			This is a note.
		\end{note}
	\end{mycolumns}
	\vfill
	Versions without captions:
	\begin{mycolumns}[columns=3]
		\begin{definition}{}
			This is a definition.
		\end{definition}
	\mynextcolumn
		\begin{example}{}
			This is an example.
		\end{example}
	\mynextcolumn
		\begin{note}{}
			This is a note.
		\end{note}
	\end{mycolumns}
	\vfill
	Tight versions with white background (e.g., for use with pictures):
	\begin{mycolumns}[columns=3]
		\begin{definitiontight}{A Definition}
			\centering\includegraphics[width=.75\linewidth]{example-image}
		\end{definitiontight}
	\mynextcolumn
		\begin{exampletight}{An Example}
			\centering\includegraphics[width=.75\linewidth]{example-image}
		\end{exampletight}
	\mynextcolumn
		\begin{notetight}{A Note}
			\centering\includegraphics[width=.75\linewidth]{example-image}
		\end{notetight}
	\end{mycolumns}
\end{frame}

{\MakeNewBox{amazing}{purple}
\begin{frame}{Create New and Modify Existing Boxes}
	\begin{mycolumns}
		\begin{amazing}{An Amazing Box}
			You can create new boxes with \texttt{\textbackslash MakeNewBox\{name\}\{color\}} (which will check if any of the new box commands is already taken) or \texttt{\textbackslash DeclareBox\{name\}\{color\}} (which may overwrite existing commands). This box was created by \texttt{\textbackslash MakeNewBox\{amazing\}\{purple\}}.
		\end{amazing}
	\mynextcolumn
		{\UpdateBoxColor{definition}{teal}
		\begin{definition}{This is a Definition}
			With \texttt{\textbackslash UpdateBoxColor\{name\}\{color\}} you can change the color of a box (locally to the current group). This definition was changed with \texttt{\textbackslash UpdateBoxColor\{definition\}\{teal\}}.
		\end{definition}}
		\bigskip
		\begin{definition}{Another Definition}
			This definition appears outside of the group, so \texttt{\textbackslash UpdateBoxColor\{definition\}\{teal\}} is no longer active.
		\end{definition}
	\end{mycolumns}
\end{frame}
}

{\MakeNewBox[Improved-Box: ]{improved}{purple}[\hfill Jeah! \boxnumber]
\begin{frame}{Counting Boxes!}
	\UpdateBoxPrefix{definition}{Definition~\boxnumber: }%
	\begin{definition}{Hey Look At My Number!}
		With \texttt{\textbackslash UpdateBoxPrefix\{name\}\{prefix\}} you can change text to appear before the title of a box (locally to the current group). The macro \texttt{\textbackslash boxnumber} refers to the number of the current box that is robust against beamer overlays (but you do not have to use it).\medskip

		This box was updated with \texttt{\textbackslash UpdateBoxPrefix\{definition\}\{Definition~\textbackslash boxnumber: \}}.
		Similarly, there is \texttt{\textbackslash UpdateBoxSuffix\{name\}\{suffix\}} for text to appear after the title. As a comfort macro, there is \texttt{\textbackslash UpdateBoxSurround\{name\}\{prefix\}\{suffix\}}.
	\end{definition}
	\bigskip
	\onslide<2->{\begin{improved}{This can be done at construction!}
		Both \texttt{\textbackslash MakeNewBox[prefix]\{name\}\{color\}[postfix]} and \texttt{\textbackslash DeclareBox[prefix]\{name\}\{color\}[postfix]} have optional arguments that allow to set the prefix and suffix at construction. This box was created with \texttt{\textbackslash MakeNewBox[Improved-Box: ]\{improved\}\{purple\}[\textbackslash hfill Jeah! \textbackslash boxnumber]}.
	\end{improved}}%
\end{frame}
}

\subsection{Adding Links to Pictures Automatically}
\begin{frame}{\insertsubsection}
	Using the command \texttt{\textbackslash pic\{filename\}} a picture can be included. In order to automatically add a (source) link to the picture, the link simply has to be stored in a txt-file with the same filename:

	\centering\pic[width=0.5\textwidth]{example-image} % omit file extension of the image picture
\end{frame}

\subsection{Automatic Dark-Mode for Pictures}
\begin{frame}{\insertsubsection}
	\begin{mycolumns}
		\begin{notetight}{Normal Image with \texttt{\textbackslash pic}}
			\centering
			\pic[width=.7\textwidth]{example-image}
		\end{notetight}
	\mynextcolumn
		\begin{notetight}{Inverted Image with \texttt{\textbackslash picDark} (in dark-mode only)}
			\centering
			\picDark[width=.7\textwidth]{example-image.jpg} % does only work for bitmap images so far
		\end{notetight}
	\end{mycolumns}
	\begin{note}{Explanation}
		Using \texttt{\textbackslash picDark} instead of \texttt{\textbackslash pic}, the dark version of an image that is saved as \texttt{<image-name>-dark} automatically gets used when dark-mode is enabled.
		
		Images can also automatically be inverted if there is no separate dark version. Therefore, white gets converted to a dark gray that matches the dark background color of the slides.
	\end{note}
\end{frame}

\section{Slide Layouts}

\subsection{Multiple Columns}
\begin{frame}{\insertsubsection}
	\begin{mycolumns}[columns=4] % default: 2 columns
		This is an example text that is shown in the \textbf{first column}.
	\mynextcolumn
		This is an example text that is shown in the \textbf{second column}.
	\mynextcolumn
		This is an example text that is shown in the \textbf{third column}.
	\mynextcolumn
		This is an example text that is shown in the \textbf{last column}.
	\end{mycolumns}
	\vfill
	\begin{note}{Explanation}
		Columns are separated by the command \texttt{\textbackslash mynextcolumn}.

		The option \texttt{columns} specifies the number of columns that should be generated. The default number of columns is two.
	\end{note}
\end{frame}

\subsection{Columns with Custom Width}
\begin{frame}{\insertsubsection}
	\begin{mycolumns}[columns=3,widths={20,30}]
		\begin{example}{}
			The \textbf{first column} takes 20\,\% of the width of the slide.
		\end{example}
	\mynextcolumn
		\begin{example}{}
			The \textbf{second column} takes 30\,\% of the width of the slide.
		\end{example}
	\mynextcolumn
		\begin{example}{}
			The \textbf{third column} takes the remaining 50\,\% of the width of the slide.
		\end{example}
	\end{mycolumns}
	\vfill
	\begin{note}{Explanation}
		The widths of the single columns can also be manually specified by a list of percentages given to the option \texttt{widths}.

		All columns whose width is not specified are equally split over the remaining space.
	\end{note}
\end{frame}

\subsection{Columns with Custom Height}
\begin{frame}{\insertsubsection}
	\begin{mycolumns}[height=6cm]
		\begin{example}{}
			This is the content of the first column.
		\end{example}
		\begin{note}{Explanation}
			If the content of a column exceeds a certain height, the common centering of all columns will not work anymore because beamer does not define the height of a slide. In this case, you can set the height of the columns manually by adding the option \texttt{height=<height>}.
		\end{note}
	\mynextcolumn
		\begin{example}{}
			This is the content of the second column. It is too high and would make the first column move downwards without additional options.
			\vspace{80mm}
		\end{example}
	\end{mycolumns}
\end{frame}

\subsection{Nested Columns}
\begin{frame}{\insertsubsection}
	\begin{mycolumns}[T]
		\begin{example}{}
			This is the content of the first column.
		\end{example}
		\begin{mycolumns}[columns=4]
			\mynote{}{1.1}
			\mynextcolumn
			\mynote{}{1.2}
			\mynextcolumn
			\mynote{}{1.3}
			\mynextcolumn
			\mynote{}{1.4}
		\end{mycolumns}
	\mynextcolumn
		\begin{example}{}
			A column on the right side
		\end{example}
		\begin{mycolumns}[T]
			\mynote{}{2.1}
			\begin{mycolumns}[columns=3]
				\mynote{}{2.1.1}
				\mynextcolumn
				\mynote{}{2.1.2}
				\mynextcolumn
				\mynote{}{2.1.3}
			\end{mycolumns}
			\mynextcolumn
			\mynote{}{2.2}
		\end{mycolumns}
	\end{mycolumns}
\end{frame}

\subsection{Column Animations}
\begin{frame}{\insertsubsection\ -- Option keep}
	\begin{mycolumns}[columns=3,keep] % short for animation=keep
		This is an example text that is shown in the \textbf{first column}.
	\mynextcolumn
		This is an example text that is shown in the \textbf{second column}.
	\mynextcolumn
		This is an example text that is shown in the \textbf{third column}.
	\end{mycolumns}
	\vfill
	\begin{note}{Explanation}
		The option \texttt{keep} (short for \texttt{animation=keep}) introduces an animation which lets the columns appear one after another.

		The previous columns are kept on the slide when a new column is displayed.

		Animations are ignored in handout mode.
	\end{note}
\end{frame}

\begin{frame}{\insertsubsection\ -- Option forget}
	\begin{mycolumns}[columns=3,forget] % short for animation=forget
		This is an example text that is shown in the \textbf{first column}.
	\mynextcolumn
		This is an example text that is shown in the \textbf{second column}.
	\mynextcolumn
		This is an example text that is shown in the \textbf{third column}.
	\end{mycolumns}
	\vfill
	\begin{note}{Explanation}
		The option \texttt{forget} (short for \texttt{animation=forget}) introduces an animation which lets the columns appear one after another.

		The previous columns are removed when a new column is displayed.

		Animations are ignored in handout mode.
	\end{note}
\end{frame}

\begin{frame}{\insertsubsection\ -- Option animation=none}
	\begin{mycolumns}[columns=3,animation=none]
		This is an example text that is shown in the \textbf{first column}.
	\mynextcolumn
		This is an example text that is shown in the \textbf{second column}.
	\mynextcolumn
		This is an example text that is shown in the \textbf{third column}.
	\end{mycolumns}
	\vfill
	\begin{note}{Explanation}
		The option \texttt{animation=none} deactivates the animation of the columns.

		In the template, this is the default. However, you can overwrite the defaults to animate all slides easily (see \texttt{setmycolumnsdefault} in the source code). Then, it might be necessary to deactivate those animations for some slides.
	\end{note}
\end{frame}

\begin{frame}[label=current]{\insertsubsection\ -- Option reverse}
	\begin{mycolumns}[columns=4,forget,reverse,b]
		\pic[width=\linewidth]{example-image}\\[0mm]~
	\mynextcolumn
		\pic[width=\linewidth]{example-image}\\[5mm]~
	\mynextcolumn
		\pic[width=\linewidth]{example-image}\\[10mm]~
	\mynextcolumn
		\pic[width=\linewidth]{example-image}\\[15mm]~
	\end{mycolumns}
	\begin{note}{Explanation}
		The option \texttt{reverse} inverts the order of any animation that is specified (i.e., using \texttt{keep} or \texttt{forget}).
	\end{note}
\end{frame}

\section{Other Features}

\subsection{Auto-Scaling for Long Titles}
\begin{frame}{\insertsubsection\ -- Long Titles are Scaled Down to the Available Space}
	\begin{note}{Explanation}
		Very long frame titles are scaled down automatically.

		\hfill\hfill\hfill This can avoid annoying linebreaks for a single character or word.\hfill~

		\hfill Use with care.
	\end{note}
\end{frame}

\subsection{Easy Navigation in Slides}
\begin{frame}{\insertsubsection}
	\vfill
	Click on the title (here \textbf{\insertshorttitle}) or subtitle (here \textbf{\insertshortsubtitle}) in the slide footer to jump to the title slide.
	\vfill
	Click on the section title (here \textbf{\insertsection}) in the slide footer to jump to the section overview.
	\vfill
	In the section overview, click on sections or subsections to jump to another part.
	\vfill
\end{frame}

\subsection{Repeating the Last Title Slide}
 % can be used as the last slide of your talk
\againtitle

\subsection{Content Overview}
% helpful to navigate to specific content, no need to show during the talk
%\contentoverview % create a content overview with one column
\contentoverview[3] % create a content overview with three columns

\end{document}